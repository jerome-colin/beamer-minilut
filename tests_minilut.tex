% Copyright 2019 by Jordi Inglada
%
% This file may be distributed and/or modified
%
% 1. under the LaTeX Project Public License and/or
% 2. under the GNU Public License.
%

\documentclass[8pt]{beamer}
\usetheme{Cesbio}
\setbeamertemplate{footnote}{%
  \parindent 0em\noindent%
  \raggedright%\raggedright
  \usebeamercolor{footnote}\hbox to 3.5em{\hfil\insertfootnotemark}\insertfootnotetext\par%
}
\setbeamertemplate{navigation symbols}{}
\usepackage[english]{babel}
\usepackage[utf8]{inputenc}
\usepackage[T1]{fontenc}
\usepackage{graphicx}
\usepackage{FiraSans}
\usepackage{mathtools}
\usepackage{ulem}

% The short title appears at the bottom of every slide, the full title is only on the title page
\title{Point d'avancement MAJA du 7 novembre 2019} 
\newcommand{\shorttitle}{branch minilut\_test}

\author{J. Colin\inst{1}}% \and Jean Bombeur\inst{2} \and Igor Le Tractor\inst{3}}
\newcommand{\shortauthor}{J. Colin} % Name to display in the footline
\institute{\inst{1}Centre d'\'Etudes Spatiales de la Biosph\`ere, France \url{jerome.colin@cesbio.cnes.fr}\\
%\inst{2}Université des \^Iles Sandwich, Royaume-Uni\\
%\inst{3}Laboratoire de mécanique agricole, Moldavie
}

\date{\today} % Date, can be changed to a custom date

\begin{document}
{
	\setbeamertemplate{footline}{} % Remove footline in titlepage
	\begin{frame}
		\vspace{.8cm}
		\includegraphics[width=1.5cm]{logo_cesbio_transp.pdf}
	\titlepage % Print the title page as the first slide
	\end{frame}
}

%----------------------------------------------------------------------------------------
%\begin{frame}
%	\frametitle{Overview}
%	\tableofcontents
%\end{frame}

%----------------------------------------------------------------------------------------
%   PRESENTATION SLIDES
%----------------------------------------------------------------------------------------

%------------------------------------------------
%\subsection{...} 
%------------------------------------------------
\begin{frame}
\frametitle{Alta Floresta, 17 Sept. 2017}
	
	\begin{columns}
		\begin{column}{0.7\textwidth}
			\begin{center}
				
		     	\includegraphics[width=1\textwidth]{figs/figs_BCXX/BC20/MosaicProduit/20170917S2A.jpg}
		     	%21LWK 2017-09-17(S2B)
		    \end{center}		
		\end{column}
		\begin{column}{0.3\textwidth}
			\begin{center}
				\begin{itemize}
					\item 21LWK 2017-09-17(S2B)
					\item Simulation avec CAMS
				\end{itemize}
		    \end{center}		
		\end{column}
	\end{columns}
\end{frame}

%------------------------------------------------
\begin{frame}
\frametitle{Alta Floresta, 17 Sept. 2017}
	\includegraphics[width=1\textwidth]{figs/figs_BCXX/BC20/log21LWK_BC20_aerosols.png}
	Proportion significative de Black Carbon ($m_{r}=1.75, m_{i}=0.20$)
	
	Organic Matter avec des RH plutôt faibles ($m_{i_{RH30}}=0.009726$, $m_{i_{RH80}}=0.00486$)
\end{frame}

%------------------------------------------------
\begin{frame}
\frametitle{Alta Floresta, 17 Sept. 2017}
	
	\begin{columns}
		\begin{column}{0.5\textwidth}
			\begin{center}
				Bande 1
		     	\includegraphics[width=1\textwidth]{figs/figs_BCXX/BC20/BC20_B1.png}
		     	%21LWK 2017-09-17(S2B)
		    \end{center}		
		\end{column}
		\begin{column}{0.5\textwidth}
			\begin{center}
				Bande 2
				\includegraphics[width=1\textwidth]{figs/figs_BCXX/BC20/BC20_B2.png}
		    \end{center}		
		\end{column}
	\end{columns}
	Miniluts : ajustements non monotones pour AOT > 1.5
\end{frame}

%------------------------------------------------
\begin{frame}
\frametitle{Deux pistes}
	\begin{columns}
		\begin{column}{0.6\textwidth}
			\textbf{Piste 1: revoir le calcul des miniluts}
			
			\begin{itemize}			
			\item Dans les spécifications :
			
			$\rho_{surf_{(b,\rho_{toa},\tau,z)}} = \sum\limits_{M=1}^n r_{M_{n}}.\rho_{surf_{M}}(b,\rho_{toa},\tau,z)$
			
			Hypothèse de linéaritée, à l'évidence non vérifiée ici
			
			\item Test :
			
			Ajout d'une pondération de $\tau$ par la proportion $r_{M_{n}}$ de chaque modèle $n$ d'aérosol $M$ tel que :
			
			$\rho_{surf_{(b,\rho_{toa},\tau,z)}} = \sum\limits_{M=1}^n r_{M_{n}}.\rho_{surf_{M}}(b,\rho_{toa},r_{M_{n}}.\tau,z)$
			\end{itemize}
		\end{column}
		\begin{column}{0.4\textwidth}
			%\begin{center}
			\textbf{Piste 2: jouer sur l'absorption du Black Carbon}
			\begin{itemize}
				\item Initialement, $m_{i}=0.45$
				\item déjà ramenée à $0.20$ dans les LUTs
			\end{itemize}

			Comparaison avec des LUTs telles que :
			\begin{itemize}
				\item $m_{i}=0.10$ (BC10)
				\item $m_{i}=0.15$ (BC15)
				\item $m_{i}=0.20$ (BC20)
			\end{itemize}
				
		\end{column}
	\end{columns}
\end{frame}

%------------------------------------------------
\begin{frame}
\frametitle{Piste 1}
	\begin{columns}
		\begin{column}{0.5\textwidth}
			\begin{center}
				Minilut B1 "spéc'"
		     	\includegraphics[width=1\textwidth]{figs/figs_BCXX/BC20/BC20_B1.png}
		    \end{center}		
		    
		\end{column}
		\begin{column}{0.5\textwidth}
			\begin{center}
				Minilut B1 "piste 1"	
				\includegraphics[width=1\textwidth]{figs/figs_minilut/21LWK_Dbg_LUTS_170917/Piste1_B1.png}
		    \end{center}
		\end{column}
	\end{columns}
\end{frame}

%------------------------------------------------
\begin{frame}
\frametitle{Piste 1}
	\begin{columns}
		\begin{column}{0.5\textwidth}
			\begin{center}
				Minilut B2 "spéc'"
		     	\includegraphics[width=1\textwidth]{figs/figs_BCXX/BC20/BC20_B2.png}
		    \end{center}		
		    
		\end{column}
		\begin{column}{0.5\textwidth}
			\begin{center}
				Minilut B2 "piste 1"	
				\includegraphics[width=1\textwidth]{figs/figs_minilut/21LWK_Dbg_LUTS_170917/Piste1_B2.png}
		    \end{center}
		\end{column}
	\end{columns}
\end{frame}

%------------------------------------------------
\begin{frame}
\frametitle{Piste 1}
	\begin{columns}
		\begin{column}{0.5\textwidth}
			\begin{center}
				Minilut B3 "spéc'"
		     	\includegraphics[width=1\textwidth]{figs/figs_BCXX/BC20/BC20_B3.png}
		    \end{center}		
		    
		\end{column}
		\begin{column}{0.5\textwidth}
			\begin{center}
				Minilut B3 "piste 1"	
				\includegraphics[width=1\textwidth]{figs/figs_minilut/21LWK_Dbg_LUTS_170917/Piste1_B3.png}
		    \end{center}
		\end{column}
	\end{columns}
\end{frame}

%------------------------------------------------
\begin{frame}
\frametitle{Piste 1}
	\begin{columns}
		\begin{column}{0.5\textwidth}
			\begin{center}
				Minilut B4 "spéc'"
		     	\includegraphics[width=1\textwidth]{figs/figs_BCXX/BC20/BC20_B4.png}
		    \end{center}		
		    
		\end{column}
		\begin{column}{0.5\textwidth}
			\begin{center}
				Minilut B4 "piste 1"	
				\includegraphics[width=1\textwidth]{figs/figs_minilut/21LWK_Dbg_LUTS_170917/Piste1_B4.png}
		    \end{center}
		\end{column}
	\end{columns}
	\begin{itemize}
		\item Problème réglé en apparence, car trop peu de correction atmosphérique
		\item Sur le run Alta Floresta, le mode backward n'arrive pas à créer un composite car les épaisseurs optiques dépassent le seuil d'$AOT_{max} = 0.5$ pour toutes les dates
	\end{itemize}
\end{frame}

%------------------------------------------------
\begin{frame}
\frametitle{Exemple sur la Crau : sortie "spéc'"}
			\begin{center}
		     	\includegraphics[width=0.75\textwidth]{figs/figs_minilut/31TFJ_Ref_MosaicProduit/20190722S2A.jpg}
		    \end{center}		
\end{frame}

%------------------------------------------------
\begin{frame}
\frametitle{Exemple sur la Crau : sortie "piste 1'"}
	\begin{center}
     	\includegraphics[width=0.75\textwidth]{figs/figs_minilut/31TFJ_Dbg_MosaicProduit/20190722S2A.jpg}
    \end{center}		
\end{frame}

%------------------------------------------------
\begin{frame}
\frametitle{Exemple sur la Crau : $r_{M}$ et $RH$}
	\includegraphics[width=1\textwidth]{figs/figs_minilut/31TFJ_Ref_LUTS_20190801/log31TFJ_201908LUT46r1_aerosols.png}

	L'implémentation "piste 1"
	\begin{itemize}
		\item génère bien un composite sur des tuiles avec moins de matières organiques et de matières carbonnées comme à la Crau (figure ci-dessus)
		\item mais ne résou pas le problème de départ...
	\end{itemize}
\end{frame}

%------------------------------------------------
\begin{frame}
\frametitle{Piste 2 : LUT avec $m_{i_{Black Carbon}} = 0.20$}
	\begin{center}
     	\includegraphics[width=0.75\textwidth]{figs/figs_BCXX/BC20/MosaicProduit/20170917S2A.jpg}
    \end{center}		
\end{frame}

%------------------------------------------------
\begin{frame}
\frametitle{Piste 2 : LUT avec $m_{i_{Black Carbon}} = 0.15$}
	\begin{center}
     	\includegraphics[width=0.75\textwidth]{figs/figs_BCXX/BC15/MosaicProduit/20170917S2A.jpg}
    \end{center}		
\end{frame}

%------------------------------------------------
\begin{frame}
\frametitle{Piste 2 : LUT avec $m_{i_{Black Carbon}} = 0.10$}
	\begin{center}
     	\includegraphics[width=0.75\textwidth]{figs/figs_BCXX/BC10/MosaicProduit/20170917S2A.jpg}
    \end{center}		
\end{frame}

%------------------------------------------------
\begin{frame}
\frametitle{Piste 2 : LUT avec $m_{i_{Black Carbon}} = 0.20$}
	\begin{columns}
		\begin{column}{0.5\textwidth}
			\begin{center}
	     		\includegraphics[width=0.75\textwidth]{figs/figs_BCXX/BC20/BC20_B1.png}	
		     	\includegraphics[width=0.75\textwidth]{figs/figs_BCXX/BC20/BC20_B3.png}
		    \end{center}		
		    
		\end{column}
		\begin{column}{0.5\textwidth}
			\begin{center}
		     	\includegraphics[width=0.75\textwidth]{figs/figs_BCXX/BC20/BC20_B2.png}
		     	\includegraphics[width=0.75\textwidth]{figs/figs_BCXX/BC20/BC20_B4.png}
		    \end{center}				
		\end{column}
	\end{columns}
\end{frame}

%------------------------------------------------
\begin{frame}
\frametitle{Piste 2 : LUT avec $m_{i_{Black Carbon}} = 0.15$}
	\begin{columns}
		\begin{column}{0.5\textwidth}
			\begin{center}
	     		\includegraphics[width=0.75\textwidth]{figs/figs_BCXX/BC15/BC15_B1.png}	
		     	\includegraphics[width=0.75\textwidth]{figs/figs_BCXX/BC15/BC15_B3.png}
		    \end{center}		
		    
		\end{column}
		\begin{column}{0.5\textwidth}
			\begin{center}
		     	\includegraphics[width=0.75\textwidth]{figs/figs_BCXX/BC15/BC15_B2.png}
		     	\includegraphics[width=0.75\textwidth]{figs/figs_BCXX/BC15/BC15_B4.png}
		    \end{center}				
		\end{column}
	\end{columns}
\end{frame}

%------------------------------------------------
\begin{frame}
\frametitle{Piste 2 : LUT avec $m_{i_{Black Carbon}} = 0.10$}
	\begin{columns}
		\begin{column}{0.5\textwidth}
			\begin{center}
	     		\includegraphics[width=0.75\textwidth]{figs/figs_BCXX/BC10/BC10_B1.png}	
		     	\includegraphics[width=0.75\textwidth]{figs/figs_BCXX/BC10/BC10_B3.png}
		    \end{center}		
		    
		\end{column}
		\begin{column}{0.5\textwidth}
			\begin{center}
		     	\includegraphics[width=0.75\textwidth]{figs/figs_BCXX/BC10/BC10_B2.png}
		     	\includegraphics[width=0.75\textwidth]{figs/figs_BCXX/BC10/BC10_B4.png}
		    \end{center}				
		\end{column}
	\end{columns}
\end{frame}

%------------------------------------------------
%
%------------------------------------------------

%\begin{frame}
%\frametitle{References}
%\footnotesize{
%\begin{thebibliography}{99} % Beamer does not support BibTeX so references must be inserted manually as below
%\bibitem[Smith, 2012]{p1} John Smith (2012)
%\newblock Title of the publication
%\newblock \emph{Journal Name} 12(3), 45 -- 678.
%\end{thebibliography}
%}
%\end{frame}

%----------------------------------------------------------------------------------------

\end{document}

